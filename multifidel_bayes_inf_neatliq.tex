\RequirePackage{fix-cm}
\RequirePackage[hyphens]{url}
\RequirePackage[final]{graphicx} % need to show figures in draft mode
\documentclass[aps,pre,nofootinbib,superscriptaddress,linenumbers,10pt, draft,tightenlines]{revtex4-1}


% Change to a sans serif font.
\usepackage{sourcesanspro}
\renewcommand*\familydefault{\sfdefault} %% Only if the base font of the document is to be sans serif
\usepackage[T1]{fontenc}
%\usepackage[font=sf,justification=justified]{caption}
\usepackage[font=sf]{floatrow}

% Rework captions to use sans serif font.
\makeatletter
\renewcommand\@make@capt@title[2]{%
 \@ifx@empty\float@link{\@firstofone}{\expandafter\href\expandafter{\float@link}}%
  {\textbf{#1}}\sf\@caption@fignum@sep#2\quad
}%
\makeatother

%\linespread{0.956}

\usepackage{listings} % For code examples
\usepackage[usenames,dvipsnames,svgnames,table]{xcolor}
\usepackage{amsmath}
\usepackage{amssymb}
\usepackage{graphicx}
\usepackage{dcolumn}
\usepackage{boxedminipage}
\usepackage[colorlinks=true,citecolor=blue,linkcolor=blue]{hyperref}
\hypersetup{
    colorlinks=true,
    linkcolor=blue,
    filecolor=magenta,      
    urlcolor=cyan,
}
 
\urlstyle{same}

\usepackage[]{microtype}
\usepackage[obeyFinal]{todonotes}
\usepackage{import}
\usepackage{setspace, siunitx, amsmath,amsfonts, adjustbox,booktabs,cleveref}
%\usepackage{caption}
\usepackage{subcaption}
\usepackage{enumitem}

\setlistdepth{20}
\renewlist{itemize}{itemize}{20}

% initially, use dots for all levels
\setlist[itemize]{label=$\cdot$}

% customize the first 3 levels
\setlist[itemize,1]{label=\textbullet}
\setlist[itemize,2]{label=--}
\setlist[itemize,3]{label=*}


\usepackage{titlesec}
\setcounter{secnumdepth}{5}


% Units
\DeclareSIUnit\Molar{\textsc{m}}


% Comments
\newcounter{comment}
\newcommand{\comment}[2][]{%
% initials of the author (optional) + note in the margin
\refstepcounter{comment}%
{%
\setstretch{0.7}% spacing
\todo[inline, color={cyan!45},size=\small]{%
\textbf{\footnotesize [\uppercase{#1}\thecomment]:}~#2}%
}}

% Start supplementary sections

\newcommand{\beginsupplement}{%
        \onecolumngrid
        \setcounter{table}{0}
        \renewcommand{\thetable}{S\arabic{table}}%
        \setcounter{figure}{0}
        \renewcommand{\thefigure}{S\arabic{figure}}%
     }

\graphicspath{{figures/}}
\floatsetup[table]{capposition=top}
%\usepackage[superscript,biblabel]{cite}
\begin{document}


%%%%%%%%%%%%%%%%%%%%%%%%%%%%%%%%%%%%%%%%%%%%%%%%%%%%%%%%%%%%%%%%%%%%%%%%%%%%%%%%
% DOCUMENT
%%%%%%%%%%%%%%%%%%%%%%%%%%%%%%%%%%%%%%%%%%%%%%%%%%%%%%%%%%%%%%%%%%%%%%%%%%%%%%%%

\title{[OUTLINE]: Parameterization of Non-Bonded Classical Mechanics Potentials for Neat Organic Liquids using a Multi-fidelity Bayesian Inference Approach}

\author{Bryce C. Manubay} 
\email{bryce.manubay@colorado.edu}
\affiliation{University of Colorado}

\author{Michael R. Shirts}
\thanks{Corresponding author}
\email{michael.shirts@colorado.edu}
\affiliation{University of Colorado}

% Date
\date{\today}

%%%%%%%%%%%%%%%%%%%%%%%%%%%%%%%%%%%%%%%%%%%%%%%%%%%%%%%%%%%%%%%%%%%%%%%%%%%%%%%%
% ABSTRACT
%%%%%%%%%%%%%%%%%%%%%%%%%%%%%%%%%%%%%%%%%%%%%%%%%%%%%%%%%%%%%%%%%%%%%%%%%%%%%%%%


\emph{Keywords: force field; molecular dynamics simulation; parameterization; inference; metamodels}

\maketitle


%%%%%%%%%%%%%%%%%%%%%%%%%%%%%%%%%%%%%%%%%%%%%%%%%%%%%%%%%%%%%%%%%%%%%%%%%%%%%%%%
% PRELIMINARIES
%%%%%%%%%%%%%%%%%%%%%%%%%%%%%%%%%%%%%%%%%%%%%%%%%%%%%%%%%%%%%%%%%%%%%%%%%%%%%%%%

\section{Preliminaries}
Definitions
\begin{itemize}
\item $V$: Volume
\item $U$: Total energy (including potential and kinetic, excluding external energy such as due to gravity, etc)
\item $S$: Entropy
\item $N$: Number of particles
\item $T$: Temperature
\item $P$: Pressure
\item $k_B$: Boltzmann constant
\item $\beta$: $(k_B T)^{-1}$
\item $M$: Molar mass
\item $\rho$: Density ($M/V$)
\item $H$: Enthalpy 
\item $G$: Gibbs Free Energy (free enthalpy)
\item $A$: Helmholtz Free Energy
\item $u$: reduced energy
\item $f$: reduced free energy
\end{itemize}
\pagebreak
%%%%%%%%%%%%%%%%%%%%%%%%%%%%%%%%%%%%%%%%%%%%%%%%%%%%%%%%%%%%%%%%%%%
%INTRODUCTION
%%%%%%%%%%%%%%%%%%%%%%%%%%%%%%%%%%%%%%%%%%%%%%%%%%%%%%%%%%%%%%%%%%%

\section{Introduction}
\begin{itemize}
	\item MD as a critical research tool
	%MD would be great, but FF issues
	%Huge amount of work to get protein relatively correct
	\begin{itemize}
		\item Force fields that are transferable and quantitatively accurate are necessary 
              for molecular simulation to be useful. \cite{villin,villin2,drug_discov}
	\end{itemize}
    \item Transferability and inaccuracy issues
    \begin{itemize}
    	\item Transferability of MD force fields, and particularly sets of force field
              parameters, is a current limitation 
    	      in the molecular simulation field.\cite{transferability1,transferability2,
    	      transferability3,transferability4}  
        \item Inaccurate and poorly parameterized force fields have been shown to grossly
              misrepresent molecular systems.
              \cite{ffcomp1,ffcomp2,robustness} 
        \item It has been shown that depending on the choice of force field, the same 
              experiments for the same or similar systems can produce quantitatively
              different results, making the choice of force field far more important than it 
              should be. \cite{ffcomp1,ffcomp2,ewen_comparison_2016,petrov_are_2014,
              guvench_comparison_2008}\\
    \end{itemize}
	\item Parameterization efforts
	\begin{itemize}
		\item Early 
		\begin{itemize}
			\item Until very recently, force fields have been parameterized manually, 
                  guided by the intuition of expert computational chemists.\cite{parm94,
                  burger,law,combined,rational,aipar,charmm1,charmm2,mm1,mm2,mmff}
			\item Despite attempts at improvement, many of the functional forms and 
                  parameters of popular force fields remain mostly unchanged due to 
                  the lack of clear, systematic methods for updating
                  them.\cite{unchanged,monticelli}
			\item Force fields like AMBER \textit{parm94} showed intuitive departure 
                  by shrinking parameter space with clever atom type 
                  defined by expert computational chemists.\cite{parm94}
		\end{itemize}
	    \item Second Gen
	    \begin{itemize}
		    \item The parameterization of GAFF used a semi-automated genetic algorithm approach 
		          to select parameters.\cite{amber}
		    \item The parameterization of the rigid Tip4p-Ew model utilized a unique gradient
                  assisted method. \cite{tip4pew}
		    \item An incredible amount of work over a long period of time has still been
                  necessary to get biomolecular force fields somewhat
		          correct.\cite{GROMOS53A5}
	    \end{itemize}
        \item Current efforts
        \begin{itemize}
        	\item A few notable attempts, such as GAAMP and ForceBalance, have been made 
                  in recent years towards the 
        	      development of more automated and systematic force field parameterization
                  methods.\cite{GAAMP,FB1,FB2,FB3} 
        	\item Each made important contributions to automated force field parameterization
                  through clever use of objective function 
        	      optimization, exploiting a variety of fitting data and allowing exploration 
                  of functional forms. 
        \end{itemize}
    \end{itemize}
    \item Bayesian parameterization
    \begin{itemize}
    	\item Previous uses
    	\begin{itemize}
    		\item Bayesian inference provides a robust statistical framework for force 
                  field parameterization. It has been shown that Bayesian approaches 
                  can be applied to a wide variety of data driven sciences.
    		      \cite{bayes1,bayes2,bayes3,bayes5,bayes6,bayes7,bayes8,bayes_coarse} 
    		%\item Bayesian methods have been used for balancing data to help minimize influence of oversampled populations and generate more robust predictive   
    		%      models\cite{bayes2} to recalibrating initial force estimates in coarse grained MD models to target atomistic MD and experimental data
    		%      \cite{bayes_coarse,bayes1}. 
    		\item Bayesian inference methods have also been applied for uncertainty 
                  quantification in MD as well as limited parameterization
    		      problems on simple Lennard-Jones systems. \cite{bayes4,UQMDrizzi,LJexpBayes}
    		% See if i can cite that unpublished MD UQ chapter from NIST
    	\end{itemize}
        \item Surrogate models/metamodels
        \begin{itemize}
            \item Parameterization on purely expensive simulator calculations (either MD or QM)
                  is extremely expensive
        	\item Metamodeling has been critical in accelerating sampling driven processes 
                  which involve expensive calculations \cite{mbar}
        	\item Some previous work has utilized efficient metamodels to accelerate Bayesian
                  inference driven parameterization of LJ models with mixed results. Statistical
                  and modeling methods were great, but the physical intuition to constrain the 
                  problem correctly was lacking. \cite{LJexpBayes}
    	    \item COFFE papers --> innovative metamodeling and sparse grid optimization 
                  techniques \cite{COFFE2016}    	    
        \end{itemize}    
        \item What our ideas for parameterization are/paper overall thesis
        \begin{itemize}
        	\item \textbf{Through systematically testing different multi-fidelity likelihood
                  estimation workflows, we have found an optimal process which 
        	      maximizes computational efficiency while yielding a force field consistent 
                  with the experimental data it was trained on.} % Not optimal, just better, it's way faster and robust 
            \item How can we combine different techniques, to find reasonable force fields 
                  in medium dimensionality in a computationally efficient manner
        \end{itemize}
        \item Additional ideas for motivating what we're doing:
        \begin{itemize}
            \item How many person-years does it typically take to create new force fields and
                  how much of that is limited by the expense of the optimization process?
        \end{itemize}
    \end{itemize}
\end{itemize}
%%%%%%%%%%%%%%%%%%%%%%%%%%%%%%%%%%%%%%%%%%%%%%%%%%%%%%%%%%%%%%%%%%%%%%%%
%METHODS
%%%%%%%%%%%%%%%%%%%%%%%%%%%%%%%%%%%%%%%%%%%%%%%%%%%%%%%%%%%%%%%%%%%%%%%%
\section{Methods}
\begin{itemize}
	\item Simulation protocol
	\item What parameters?
	\begin{itemize}
		\item Non-bonded for cyclohexane and ethanol %cyclopropane? What other molecules could add? Brainstorm list fitting
		                                             % SMIRKS we have. When we do the 10 dimension test we'll have to add more 
							     % molecules to avoid overfitting.
		%\item I'm going to add a few more molecules,
		%      but try to keep the number of parameters capped at 10 (chain alkanes, cyclic alcohols, etc.)
		% Need to decide which molecules
		\item 10
		\item Specific SMIRKS:
		\begin{itemize}
			\item $[\#8X2H1+0:1]$, $[\#6X4:1]$, $[\#1:1]-[\#6X4]$, $[\#1:1]-[\#8]$, 
			      $[\#1:1]-[\#6X4]-[\#7,\#8,\#9,\#16,\#17,\#35]$
		\end{itemize}
	\end{itemize}
    \item Property calculation
    \begin{itemize}
        \item For this paper we will be optimizing our parameters on two thermophysical
              properties; molar volume, $\hat{V}$, and heat of 
              vaporization, $\Delta H_{vap}$. This section details the methods we will use to
              calculate each.
    % BCM: Comment out some of this because we're really only going to be calculating properites one way
    % BCM: may want to use molar volumes instead of densities so don't have to propogate simulation error to new property.
    %      Easier to propogate error in experiment and transform experimental property
        \begin{itemize}
    	    \item Molar Volume \\*
    	    System volume, $V$, can easily be calculated as:
    	    \begin{equation} V = x \times y \times z \end{equation}
    	    where $x$, $y$ and $z$ are the edge lengths of the simulation.
    	    This can be converted to a molar volume by dividing by the number of moles in the     
            periodic box. We can write molar volume as:
    	    \begin{equation} \hat{V} = \frac{V}{N_{mol}} = \frac{V \times N_{Av}}{N_{part}}                   \end{equation}
    	    Where $N_{mol}$ are the number of moles per box, $N_{part}$ are the number of        
            particles per box and $N_{Av}$ 
    	    is Avogadro's number.
    	
    	%------------------------------------------------------------------------------------------------------------------------
    	\subsubsection{Heat of Vaporization}
    	% Can calculate covariances using something callable in MBAR, but it's not documented. Can also do with bootstrap
    	% and we know that's correct until we can validate the MBAR part. Can also use fluctuation calculations and bootstrap 
    	% which will probably be the absolute easiest, but te MBAR estimates of uncertainty won't be good

        The definition of the enthalpy of vaporization is:
        \begin{equation}\Delta H_{vap} = H_{gas} - H_{liq} = E_{gas} - E_{liq} + P(V_{gas} - V_{liq})\end{equation}\\*
	
    	The uncertainty in this calculation can be computed by bootstrapping or analytical
        estimation using MBAR. We will compare both results in order to determine whether 
        the cheaper analytical estimate is accurate enough to be used.\\*  

       %--------------------------------------------------------------------------------------------------------------------------
        \end{itemize}
    \end{itemize}
    \item Methods for metamodeling
    \begin{itemize}
    	\item MBAR
    	\item Surrogate models
    	\begin{itemize}

            \item GP models
            % Expand for our specific case. What precise cov matrices are we calculating?
            \begin{itemize}
            	\item Formalism for estimating some quantity $Z$ at
            	      unknown location $x_0$ ($Z\left(x_0\right)$) 
            	      from N pairs of observed values 
            	      $w_i\left(x_0\right)$ and $Z\left(x_i\right)$ where
            	      $i = 1,...,N$
                \item \begin{equation} \hat{Z}\left(x_0\right) =
                      \sum_{i=1}^N w_i\left(x_0\right) \times
                      Z\left(x_i\right) \end{equation}
                \item We find our weight matrix, \textbf{W}, by minimizing \textbf{W} subject 
                      to the following system of equations:
                \begin{itemize}
                	\item \begin{align}
                	&\underset{W}{\text{minimize}}& & W^T \cdot \operatorname{Var}_{x_i} \cdot W - \operatorname{Cov}_{x_ix_0}^T \cdot W - W^T \cdot \operatorname{Cov}_{x_ix_0} + \operatorname{Var}_{x_0} \\
                	&\text{subject to}
                	& &\mathbf{1}^T \cdot W = 1
                	\end{align}
                	\item where the literals \begin{equation}\left\{\operatorname{Var}_{x_i},
                          \operatorname{Var}_{x_0},                    
                          \operatorname{Cov}_{x_ix_0}\right\}\end{equation} stand for
                          \begin{equation}
                	\left\{\operatorname{Var}\left(\begin{bmatrix}Z(x_1)&\cdots&Z(x_N)\end{bmatrix}^T\right), \operatorname{Var}(Z(x_0)), \operatorname{Cov} \left(\begin{bmatrix}Z(x_1)&\cdots&Z(x_N)\end{bmatrix}^T,Z(x_0)\right)\right\}\end{equation}
                \end{itemize}
                \item The weights summarize important procedures of the 
                      inference process:    	
                \begin{itemize}
                	\item They reflect the structural closeness of 
                	      samples to the estimation location, $x_0$
                	\item They have a desegregating effect, to avoid
                	      bias caused by sample clustering
                \end{itemize}
                \item Formalizing expressions for my specific problem (which variables are which
                      and what exactly do the covariances refer to).
            \end{itemize}
    	\end{itemize}
        \item \textbf{Hypothesis: With a multi-fidelity hierarchical observable calculation
               scheme, we can quickly approach the true forward model produced 
        	   by MD simulation}
        \item Explanation of potential multi-fidelity posterior sampling algorithms:
        \begin{itemize}
        	\item 3 Levels of property calculation
        	\begin{itemize}
        		\item High fidelity: Full MD simulation at a single point in parameter space
        		\item Medium fidelity: Use MBAR to estimate properties over a conservative 
                      range of parameter space in order to create a hypervolume of data
        		      over which we can construct a model.
        		\item Low fidelity: Use data from medium fidelity calculations in order to fit
                      a regression model over a hypervolume of parameter space
        		\begin{itemize}
        			\item For right now, most plausible technique is GP regression, but 
                          could brainstorm some others
        		\end{itemize}
            \end{itemize}    
            \item Sampling from our posterior using GP regression will rapidly lead us to a new 
                  optima in the local parameter space that we have modeled.
            \item We can directly sample configurations at this new optima using MD and 
                  repeat the hierarchical property calculation and sampling using GP
                  models in order to "learn" the functional form of the properties
                  across parameter space.
            \item We arrive in the high probability region of our parameters cheaply and can 
                  finish sampling using the high fidelity calculation method. 
%            \item Using MBAR as a look up table (2 levels of property estimation)
%            \begin{itemize}
%            	\item High fidelity: Full MD simulation at a single point in parameter space
%            	\item Medium fidelity: Use MBAR to estimate properties over a conservative range of parameter space in order to create a hypervolume of data
%            	\begin{itemize}
%            		\item Rather than attempting to fit a model we can use MBAR discrete MBAR calculated observables in order to evaluate our likelihood
%            		\item Using a discrete set of calculations, we can iterate over those values in order to find the highest point of probability and then 
%            		      perform a new simulation and repeat the process
%            		\item As we narrow in on the region of highest probability density, we can refine the grid over which we're searching in order to more
%            		      accurately represent the final posterior
%            % Could be useful in low dimensionality and to start with
%            	\end{itemize}
%            \end{itemize}
        \end{itemize}
    \end{itemize}
%%%%%%%%%%%%%%%%%%%%%%%%%%%%%%%%%%%%%%%%%%%%%%%%%%%%%%%%%%%%%%%%%%%%%%%%%
% Validating accuracy of MBAR calculations for constructing GP models
%%%%%%%%%%%%%%%%%%%%%%%%%%%%%%%%%%%%%%%%%%%%%%%%%%%%%%%%%%%%%%%%%%%%%%%%%
% 1) Discussions of N_eff and some of the work by Rich
% 2) Parity plots between simulation and MBAR
%%%%% a) Overlays with N_eff heat maps
%%%%% b) Observable calculations and their error
% 3) Is there a generalizable relationship between N_eff and reliability of MBAR calculations?
%%%%% a) Extendable across systems? (test another with ethanol)
%%%%% b) Extendable across SMIRKS? (test the other SMIRKS present in cyclohexane)
% 4) 
\end{itemize}
%%%%%%%%%%%%%%%%%%%%%%%%%%%%%%%%%%%%%%%%%%%%%%%%%%%%%%%%%%%%%%%%%%%%%%%%%
%PROPOSED EXPERIMENTS AND ANALYSES
%%%%%%%%%%%%%%%%%%%%%%%%%%%%%%%%%%%%%%%%%%%%%%%%%%%%%%%%%%%%%%%%%%%%%%%%%
\section{Experiments + Results and Analyses}
\begin{itemize}
	\item \textbf{Hypothesis: Using a multi-fidelity likelihood calculation scheme described in the previous section will provide not only a substantial
	      speed up over a traditional inference approach with purely simulation used in the likelihood estimate, but 
	      will also result in statistically equivalent final force field.}
	\item Experiments for testing sampling workflow
        \item Results from 2D experiments with $[\#6X4:1]$
        \begin{itemize}
            \item Stability of solution (multiple starting points) %using final process
            \begin{itemize}
                \item Maybe also consider does starting out of liquid phase break the process 
            \end{itemize}
            \item How similar do all of the final posteriors end up being %Determine a way to measure closeness
            \item Simulation of properties using final distribution of parameters
            \item How does sparsity of MBAR calculations in GP models affect final posterior?
            \begin{itemize}
                \item Start with gradient information (2 $\times$ dimensionality + 1) and move up
                      from there. %Add in all correlation effects because it won't be that expensive. Discussion of higher dimension correlation further out
                \item What is the most dense the calculations should be and how many factor
                      levels should be tested (this may require some trial-and-error to sort)?
            \end{itemize}
            \item Checking across 3 general small molecule force fields (\textit{GAFF2},
                  \textit{parm99} and \textit{CGenFF}) the standard 
                  deviation in $\epsilon$ was ~0.0034 and in
                  r$_{min,half}$ was ~0.0638, which are about 21.7 and 4.3 \% of the 
                  \textit{smirnoff99Frosst} values, respectively.
            \item Given standard deviations of parameters from other general organic forcefields,
                  the stability tests will be, at max, 20 \% change in $\epsilon$ and 5 \% 
                  in r$_{min,half}$.
        \end{itemize}        
        \item Same results from 4D experiments with $[\#1:1]-[\#6X4]$ added
        \begin{itemize}
            \item Checking across 3 general small molecule force fields (\textit{GAFF2},
                  \textit{parm99} and \textit{CGenFF}) the standard deviation in $\epsilon$ 
                  was ~0.0144 and in $r_{min,half}$ was ~0.0672, which are about 13.2 and 3.5
                  \% of the \textit{smirnoff99Frosst} values, respectively. 
            \item Given standard deviations of parameters from other general organic 
                  forcefields, the stability tests will be, at max, 15 \% change in 
                  $\epsilon$ and 5 \% in r$_{min,half}$.
        \end{itemize}
        \item Scale up to all 5 SMIRKS (10-D parameter space)
        \begin{itemize}
            \item Similar variance analyses for $\epsilon$ and r$_{min,half}$ of 
                  $[\#8X2H1+0:1]$, $[\#1:1]-[\#8]$, 
                  $[\#1:1]-[\#6X4]-[\#7,\#8,\#9,\#16,\#17,\#35]$ yield the following:
            \begin{itemize}
                \item $[\#8X2H1+0:1]$: $Var\left(\epsilon\right)^{0.5} = 0.0516$ and 
                      $Var\left(r_{min,half}\right)^{0.5} = 0.0405$, which are about 
                      24.5 and 2.4 \% of the \textit{smirnoff99Frosst} values, respectively.
                \item $[\#1:1]-[\#8]$: $Var\left(\epsilon\right)^{0.5} = 0.1202$ and 
                      $Var\left(r_{min,half}\right)^{0.5} = 0.1047$, which are about 
                      40.1 and 20000 \% of the \textit{smirnoff99Frosst} values, respectively.
                      Given that these values are pretty ridiculous, I'll keep with the trend
                      of ~20 \% changes in $\epsilon$ and ~5 \% changes in r$_{min,half}$.
                \item $[\#1:1]-[\#6X4]-[\#7,\#8,\#9,\#16,\#17,\#35]$:                                                   $Var\left(\epsilon\right)^{0.5} = 0.00505$ and                                                   $Var\left(r_{min,half}\right)^{0.5} = 0.0275$, which are about 
                      32.1 and 2.0 \% of the \textit{smirnoff99Frosst} values, respectively.
            \end{itemize}
            \item Given results from variance analyses over all 5 SMIRKS/atom types, I think
                  maximum changes of 25 \% of the $\epsilon$ and 5 \% of the r$_{min,half}$ in 
                  \textit{smirnoff99Frosst} would be appropriate for the stability tests.
        \end{itemize}
    \item Ideas for comparing force fields (different start points) 
    % Robustness of parameter choices given a final posterior distribution
    \begin{itemize}
    	\item KL divergence/some other probability distribution convergence test
        \begin{itemize}
            \item Stability of solution (from different starting points do we get same answer?)
            \item Maximum likelihood calculation of GP
            \begin{itemize}
                \item More detail. Is this possible? 
            \end{itemize}
            \item Discrete KL divergence
            \item Maximum Mean Discrepancy estimators
            \item Some theory: \url{http://papers.nips.cc/paper/3417-estimation-of-information-
                  theoretic-measures-for-continuous-random-variables}
            \item Practically: \url{https://github.com/gregversteeg/NPEET}
        \end{itemize}
    	\item Speed of convergence
        \begin{itemize}
            \item Number of iterations to convergence
            \item Total wall time
        \end{itemize} 
    	\item Simulation of properties, using final parameters, that 
    	      were not in training set % Let's do same properties different molecules outside of training set
	                               % Maybe also add different temperatures for training 
    	\begin{itemize}
    		\item 3-fold verification
    		\begin{itemize}
    			%\item Different properties not in training set
    			\item Extrapolation to thermodynamic state outside of training set (T, P)
    			\item Other molecules outside of training set that have the same SMIRKS types
    		\end{itemize}    		
    	\end{itemize} 
    \end{itemize}
\end{itemize}
%%%%%%%%%%%%%%%%%%%%%%%%%%%%%%%%%%%%%%%%%%%%%%%%%%%%%%%%%%%%%%%%%%%%%%%%%%
%%%%%%%%%%%%%%%%%%%%%%%%%%%%%%%%%%%%%%%%%%%%%%%%%%%%%%%%%%%%%%%%%%%%%%%%%%%%%%%%
% BIBLIOGRAPHY
%%%%%%%%%%%%%%%%%%%%%%%%%%%%%%%%%%%%%%%%%%%%%%%%%%%%%%%%%%%%%%%%%%%%%%%%%%%%%%%%

\bibliographystyle{achemso} 
\bibliography{bayes1_manuscript}

\end{document}
